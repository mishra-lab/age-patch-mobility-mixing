\section{Introduction}
Contact patterns by age and geography mediate the dynamics of directly-transmitted infectious diseases in humans.
Arenas et al.\ (2020) developed an approach to simulate contact patterns associated with
age and recurrent mobility between geographic patches, e.g. for work and school, with three important assumptions.
We sought to build on this approach by relaxing these assumptions (make the assumptions optional),
and to examine the potential influence of each assumption on simulated contact patterns.
\section{Methods}
In our proposed approach:
age mixing patterns from the \textsc{polymod} study are maintained separate by contact type
(household, work, school, other),
versus aggregated overall (assumption~1),
so that changes to the numbers of different types of contacts are reflected in overall age mixing patterns;
age mixing patterns are demographically adjusted to each mixing pool,
versus fixed for the whole population (assumption~2),
so that contacts in the model are balanced; and
non-mobile individuals can be fully isolated from mobile travellers to their geographic patch,
versus isolation from travellers being impossible (assumption~3),
so that various interventions can be explored.
We compared contact matrices for Ontario, Canada,
using \textsc{polymod} and cellphone mobility data during the \textsc{covid-19} pandemic (reduced mobility)
before versus after relaxing each assumption.
\section{Results}
Relaxing assumption~1 led to
household contact patterns being more dominant overall in the context of reduced mobility.
Relaxing assumption~2 led to
more contacts with more populous age strata, and fewer contacts with less populous age strata,
for all contact types.
Relaxing assumption~3 led to
a greater proportion of contacts with residents of the same geographic patch for non-household contact types,
and no change to household contacts.
\section{Significance}
In the context of risk heterogeneity by age and/or geography,
assumptions used to generate contact patterns yield different inferences about mixing,
and could therefore influence model-based transmission projections and intervention assessments.