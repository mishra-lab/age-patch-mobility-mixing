\section{Introduction}
Contact patterns by age and geography are often key determinants of infectious disease dynamics in humans.
Arenas et al.\ (2020) develop a method to simulate contact patterns associated with
age and recurrent mobility between geographic patches, such as for work and school.
However, their approach makes several assumptions which may not always be suitable.
We sought to improve the approach to make those assumptions optional,
and examine the potential influence of each assumption on simulated contact patterns.
\section{Methods}
In our proposed approach:
age mixing patterns from the \textsc{polymod} study are maintained separate by contact type
(household, work, school, other),
versus aggregated overall (assumption~1),
so that changes to the numbers of different types of contacts are reflected in overall age mixing patterns;
age mixing patterns are demographically adjusted to each mixing pool,
versus fixed for the whole population (assumption~2),
so that contacts in the model are balanced; and
non-mobile individuals can be fully isolated from mobile travellers to their patch,
versus isolation from travellers being impossible (assumption~3),
so that various intervention options can be explored.
We compared contact matrices for Ontario, Canada under reduced mobility
before versus after relaxing each assumption.
\section{Results}
Relaxing assumption~1 led to
household contact patterns being more dominant overall in the context of reduced mobility.
Relaxing assumption~2 led to
more contacts with more populous age strata, and fewer contacts with less populous age strata,
for all contact types.
Relaxing assumption~3 led to
a greater proportion of contacts with residents of the same patch for non-household contact types,
and no change to household contacts.
\section{Significance}
Precise assumptions about contact patterns
are important for parameterizing transmission models stratified by age and geography,
and essential for using those models to answer research questions
related to differential interventions and/or outcomes by age and/or geography.