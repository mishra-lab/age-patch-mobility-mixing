Infectious disease transmission models often stratify populations by age and geographic patches.
Contact patterns between age groups and patches are key parameters in such models.
Arenas et al.\ (2020) develop an approach to simulate contact patterns associated with
recurrent mobility between patches, such as due to work, school, and other regular travel.
% JK: @SM added next sentence based on suggestion in 3.2 Results.
%     I have mixed feelings about it, because I think is an important point,
%     but might slightly interrupt the flow of ideas, and not sure where else to put it.
% SM: I like including this too. it didn't feel like an interruption in flow. made a few edits for consideration to suggest this point is also part of the 'results' of the paper (vs. what has already been shown?)
First, we demonstrate that using their approach, mixing between patches is greater than mobility data alone would suggest,
because individuals from patches A and B can form a contact if they meet in patch C.
We then build upon their approach to address three potential gaps that remain.
First, our approach includes a distribution of contacts by age
that is responsive to underlying age distribution of the mixing pool.
Second, different age distributions by contact type are also maintained in our approach,
such that changes to the numbers of different types of contacts
are appropriately reflected in changes to the overall age mixing patterns.
Finally, we introduce and distinguish between two mixing pools associated with each patch,
with possible implications for the overall connectivity of the population:
the home pool, in which contacts can only be formed with other individuals residing in the same patch;
and the travel pool, in which contacts can be formed with some residents of, and any other visitors to the patch.
We describe in detail the steps required to implement our approach,
and present results of an example application.
