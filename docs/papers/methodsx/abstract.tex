Infectious disease transmission models often stratify populations by age and geographic patches.
Contact patterns between age groups and patches are key parameters in such models.
Arenas et al.\ (2020) develop an approach to simulate contact patterns associated with
recurrent mobility between patches, such as due to work, school, and other regular travel.
% JK: @SM added next sentence based on suggestion in 3.2 Results.
%     I have mixed feelings about it, because I think is an important point,
%     but might slightly interrupt the flow of ideas, and not sure where else to put it.
% SM: I like including this too. it didn't feel like an interruption in flow.
%     made a few edits for consideration to suggest this point is also part of the 'results' of the paper
%     (vs. what has already been shown?)
% JK: I don't know if we've really explicitly shown this in the results
%     (which I think would require some kind of quantification) only commented it in the methods.
%     So, hopefully it's okay to revert to just below removing "First, we demonstrate that"?
Using their approach, mixing between patches is greater than mobility data alone would suggest,
because individuals from patches A and B can form contacts if they meet in patch C.
We build upon their approach to address three potential gaps that remain, outlined in the bullets below.
% JK: Update: the journal commented:
%     "The abstract should be non-standard format must contain a paragraph of text with 200 words
%     AND then UP to 3 bullet points highlighting the customization rather than the steps of the procedure."
%     I thought the bullets were optional as recent papers have not always includede it...
%     I've moved the 3 changes we originally described in the abstract into those bullets (highlights.tex)
%     and added a little but more detail to the abstract here with the available room. %SM: got it! reads well
We describe the steps required to implement our approach in detail,
and present step-wise results of an example application
as potential parameters for \sarscovii transmission modelling in Ontario, Canada. %SM: we don't do SARS modeling here, so maybe frame as application to generate parameters...
We also provide methods for deriving the mobility matrix based on GPS mobility data (appendix). %SM: define GPS or spell it out in full here since not using acronym again in abstract?
