\section{Introduction}\label{intro}
Contact patterns are key determinants of epidemic dynamics
because they define who can be infected, by whom, and how quickly \cite{Mossong2008}.
% HM: why it is the key determinant? I think this paper set the rational well:
% https://journals.plos.org/plosone/article?id=10.1371/journal.pone.0173411
% JK: I've added a bit -- but would prefer to cite a "bigger picture" paper here
% e.g. the original POLYMOD study Mossong2008, versus one applied study
\citet{Arenas2020} develop a patch-based model of \covid transmission applied to Spain,
in which the modelled population is stratified by geographic patches and three age groups.
Following foundational work by \citet{Balcan2011,Sattenspiel1995},
the model incorporates data on short, recurrent mobility patterns
to determine contact rates between individuals in different patches and age groups.
We build upon this contact model to incorporate improved age mixing patterns,
which are stratified by different contact types and are responsive to
the age distributions of mixing populations, as proposed by \citet{Arregui2018}.
% HM: why we want to include age mixing? maybe include more rational for build on top Arenas et al. model
% Arenas2020 already do include limited age mixing,
% whereas we are allowing it to be responsive to the mixing populations,
% but I've rephrased to make the rationale more obvious.
% It's hard to get into the specific without adding a lot of detail,
% which we do in in 2.1 anyways
% MH: Again - believe this should be at the beginning of the sentence.
% JK: But Arregui2018 didn't propose to "build upon this model" since it was published prior.
% hopefully it still reads okay like this?
We also explore some practical challenges in parameterizing such models.