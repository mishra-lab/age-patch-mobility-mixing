\section{Conclusion}\label{conc}
\citet{Arenas2020} develop an approach to modelling
contact patterns associated with recurrent mobility,
which is relevant to dynamic models of infectious disease transmission, such as \covid.
The original approach simulates contact patterns between
age groups and geographic patches connected by recurrently mobile individuals,
and considers changes to mixing between patches due to reduced mobility among some individuals.
However, the method could be improved to:
ensure contact balancing between age groups;
model changes to age contact patterns in response to reduced mobility; and
allow complete isolation of non-mobile individuals from mobility-related contacts.
We propose a modification of the approach that achieves these improvements.
\par
The first key change in the proposed approach draws on \cite{Arregui2018} to
combine preferential patterns of age mixing with the age distribution of the mixing population,
such that the actual number of contacts formed reflects both elements.
This change is incorporated into each separate mixing ``pool'' where contacts are formed,
and ensures that the number of contacts simulated from age group $a$ to age group $a'$
will equal those from age group $a'$ to age group $a$.
\par
The second key change in the proposed approach is to
maintain separate mixing patterns for each type of contact,
only aggregating the contribution of different contact types to overall transmission
within the force of infection equation.
With this change, the age mixing patterns associated with any contact type
are not influenced by changes to the numbers or mixing patterns of any other contact type.
This change also supports differential probability of transmission by contact type.
\par
The final key change in the proposed approach is to
introduce two separate mixing pools where contacts can form.
% MH: Think this is an improper use of a colon here.
Within ``home'' pools, contacts can only be formed with other residents of the same patch.
Within ``travel'' pools, contacts cab be formed with other residents of the same patch
who are mobile within their residence patch, or with any mobile visitors to the patch.
Home pools therefore allow true isolation of some individuals from mobility-related contacts,
with implications for overall network connectivity.
\par
In developing and applying the proposed approach to an example context,
we present the methodological details and results of each intermediate step,
so that they may be reproduced or built upon in future work.
